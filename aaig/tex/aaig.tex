% -*- ess-noweb-default-code-mode: gap-mode; -*-% ===> this file was generated automatically by noweave --- better not edit it
\documentclass[titlepage,sfsidenotes,nols]{tufte-book}

\usepackage{fontspec}
\setmonofont{Iosevka}

\usepackage{noweb}

\usepackage{color}
% https://commons.wikimedia.org/wiki/File:Erlang_logo.svg
\definecolor{ErlangRed}{HTML}{A90533}

\hypersetup{
  bookmarks=true,
  pdffitwindow=true,
  pdfstartview={FitH},
  pdftitle={Abstract Algebra in GAP: Exercises},
  pdfauthor={Eric Bailey <eric@ericb.me>},
  pdfsubject={Abstract Algebra in GAP},
  pdfkeywords={abstract algebra, GAP, literate programming, noweb},
  colorlinks=true,
  linkcolor=ErlangRed,
  urlcolor=ErlangRed
}

\usepackage{amsmath}
\usepackage{amssymb}
\usepackage{mathtools}
\mathtoolsset{centercolon}

\usepackage[outputdir=tex]{minted}

% NOTE: Use Tufte instead of noweb page style.
% \pagestyle{noweb}
% NOTE: Use shift option for wide code.
% \noweboptions{smallcode,shortxref,webnumbering,english}
\noweboptions{smallcode,shortxref}


\title{Abstract Algebra in GAP}

\author{Eric Bailey}

% \date{October 1, 2017}

\newcommand{\stylehook}{\marginpar{\raggedright\sl Style hook}}

\newmintinline[gap]{gap}{}

\begin{document}
\maketitle
\nwfilename{aaig.nw}\nwbegindocs{1}\nwdocspar

\tableofcontents
\newpage

\chapter{Basic System Interaction}

\section{Exercise 1}

\begin{enumerate}[a]

\item {\Tt{}\nwlinkedidentq{IsPerfect}{NWVU3sy-2xVIrD-1}\nwendquote} is a function that takes a positive integer \gap{n} and
returns \gap{true} if \gap{n} is perfect and \gap{false} otherwise.

We could define a function to compute the aliquot sum of a positive integer $n$:
\marginnote[\baselineskip]{%
  \[
    s(n) \equiv \sigma(n) - n
  \]
}
\nwenddocs{}\nwbegincode{2}\sublabel{NWVU3sy-xApnh-1}\nwmargintag{{\nwtagstyle{}\subpageref{NWVU3sy-xApnh-1}}}\moddef{Compute the aliquot sum of a positive integer~{\nwtagstyle{}\subpageref{NWVU3sy-xApnh-1}}}\endmoddef\nwstartdeflinemarkup\nwenddeflinemarkup
\nwindexdefn{\nwixident{AliquotSum}}{AliquotSum}{NWVU3sy-xApnh-1}\nwlinkedidentc{AliquotSum}{NWVU3sy-xApnh-1} := n -> Sum(DivisorsInt(n)) - n;
\nwnotused{Compute the aliquot sum of a positive integer}\nwidentdefs{\\{{\nwixident{AliquotSum}}{AliquotSum}}}\nwendcode{}\nwbegindocs{3}\nwdocspar

Then, using that definition, we could write a function to determine whether a
positive integer $n$ is perfect:
\nwenddocs{}\nwbegincode{4}\sublabel{NWVU3sy-3gU79I-1}\nwmargintag{{\nwtagstyle{}\subpageref{NWVU3sy-3gU79I-1}}}\moddef{Determine whether a positive integer is perfect~{\nwtagstyle{}\subpageref{NWVU3sy-3gU79I-1}}}\endmoddef\nwstartdeflinemarkup\nwenddeflinemarkup
\nwlinkedidentc{IsPerfect}{NWVU3sy-2xVIrD-1} := n -> n = \nwlinkedidentc{AliquotSum}{NWVU3sy-xApnh-1}(n);
\nwnotused{Determine whether a positive integer is perfect}\nwidentuses{\\{{\nwixident{AliquotSum}}{AliquotSum}}\\{{\nwixident{IsPerfect}}{IsPerfect}}}\nwindexuse{\nwixident{AliquotSum}}{AliquotSum}{NWVU3sy-3gU79I-1}\nwindexuse{\nwixident{IsPerfect}}{IsPerfect}{NWVU3sy-3gU79I-1}\nwendcode{}\nwbegindocs{5}\nwdocspar

Conveniently, GAP ships with \gap{Sigma}, which we can use instead.
\marginnote[-1\baselineskip]{%
  \begin{align*}
    \sigma(n) &= \sum_{d|n} d \\
    {\Tt{}\nwlinkedidentq{IsPerfect}{NWVU3sy-2xVIrD-1}\nwendquote}(n) := \sigma(n) &= 2n
  \end{align*}
}
\nwenddocs{}\nwbegincode{6}\sublabel{NWVU3sy-1Ey6ln-1}\nwmargintag{{\nwtagstyle{}\subpageref{NWVU3sy-1Ey6ln-1}}}\moddef{Determine whether a positive integer is perfect, using Sigma~{\nwtagstyle{}\subpageref{NWVU3sy-1Ey6ln-1}}}\endmoddef\nwstartdeflinemarkup\nwusesondefline{\\{NWVU3sy-2xVIrD-1}}\nwenddeflinemarkup
n -> Sigma(n) = 2*n
\nwused{\\{NWVU3sy-2xVIrD-1}}\nwendcode{}\nwbegindocs{7}\nwdocspar

\item To find all perfect numbers less than $1000$, run the following:
\marginnote[2\baselineskip]{%
  \[
    \left\{n \in \mathbb{Z}^+\ |\ 1 \leq n < 1000,\ {\Tt{}\nwlinkedidentq{IsPerfect}{NWVU3sy-2xVIrD-1}\nwendquote}(n)\right\}
  \]
}

\nwenddocs{}\nwbegincode{8}\sublabel{NWVU3sy-h8dy6-1}\nwmargintag{{\nwtagstyle{}\subpageref{NWVU3sy-h8dy6-1}}}\moddef{Find all perfect numbers less than 1000~{\nwtagstyle{}\subpageref{NWVU3sy-h8dy6-1}}}\endmoddef\nwstartdeflinemarkup\nwusesondefline{\\{NWVU3sy-2xVIrD-1}\\{NWVU3sy-2W1Ke-1}}\nwenddeflinemarkup
Filtered([1..999], \nwlinkedidentc{IsPerfect}{NWVU3sy-2xVIrD-1});
\nwused{\\{NWVU3sy-2xVIrD-1}\\{NWVU3sy-2W1Ke-1}}\nwidentuses{\\{{\nwixident{IsPerfect}}{IsPerfect}}}\nwindexuse{\nwixident{IsPerfect}}{IsPerfect}{NWVU3sy-h8dy6-1}\nwendcode{}\nwbegindocs{9}\nwdocspar

... which results in:
\nwenddocs{}\nwbegincode{10}\sublabel{NWVU3sy-1gKlN7-1}\nwmargintag{{\nwtagstyle{}\subpageref{NWVU3sy-1gKlN7-1}}}\moddef{All perfect numbers less than 1000~{\nwtagstyle{}\subpageref{NWVU3sy-1gKlN7-1}}}\endmoddef\nwstartdeflinemarkup\nwusesondefline{\\{NWVU3sy-2xVIrD-1}\\{NWVU3sy-2W1Ke-1}}\nwenddeflinemarkup
[ 6, 28, 496 ]
\nwused{\\{NWVU3sy-2xVIrD-1}\\{NWVU3sy-2W1Ke-1}}\nwendcode{}\nwbegindocs{11}\nwdocspar

\item Not all numbers of the form $2^n(2^{n+1} - 1)$, for some positive integer
  $n$, are perfect.

\nwenddocs{}\nwbegincode{12}\sublabel{NWVU3sy-2uZKeV-1}\nwmargintag{{\nwtagstyle{}\subpageref{NWVU3sy-2uZKeV-1}}}\moddef{Not all perfect~{\nwtagstyle{}\subpageref{NWVU3sy-2uZKeV-1}}}\endmoddef\nwstartdeflinemarkup\nwenddeflinemarkup
gap> ForAll( PositiveIntegers,
>            n -> \nwlinkedidentc{IsPerfect}{NWVU3sy-2xVIrD-1}(2^n * (2^(n+1) - 1)) );
false
\nwnotused{Not all perfect}\nwidentuses{\\{{\nwixident{IsPerfect}}{IsPerfect}}}\nwindexuse{\nwixident{IsPerfect}}{IsPerfect}{NWVU3sy-2uZKeV-1}\nwendcode{}\nwbegindocs{13}\nwdocspar

\item In Euclid's formation rule (IX.36), he proved $\frac{q(q+1)}{2}$ is an
  even perfect number where $q$ is a prime of the form $2^p - 1$ for prime $p$,
  a.k.a. a Mersenne prime.

\nwenddocs{}\nwbegincode{14}\sublabel{NWVU3sy-hWGw1-1}\nwmargintag{{\nwtagstyle{}\subpageref{NWVU3sy-hWGw1-1}}}\moddef{Euclid's IX.36~{\nwtagstyle{}\subpageref{NWVU3sy-hWGw1-1}}}\endmoddef\nwstartdeflinemarkup\nwenddeflinemarkup
gap> MersennePrimes := Filtered( List( Primes\{[1..50]\},
                                       p -> 2^p - 1 ),
                                 IsPrime );
[ 3, 7, 31, 127, 8191, 131071, 524287, 2147483647,
  2305843009213693951, 618970019642690137449562111,
  162259276829213363391578010288127,
  170141183460469231731687303715884105727 ]
gap> ForAll( MersennePrimes, q -> \nwlinkedidentc{IsPerfect}{NWVU3sy-2xVIrD-1}(q * (q + 1) / 2) );
true
\nwnotused{Euclid's IX.36}\nwidentuses{\\{{\nwixident{IsPerfect}}{IsPerfect}}}\nwindexuse{\nwixident{IsPerfect}}{IsPerfect}{NWVU3sy-hWGw1-1}\nwendcode{}\nwbegindocs{15}\nwdocspar

\item TODO: Prove it.

\end{enumerate}


\subsection{Code}

\nwenddocs{}\nwbegincode{16}\sublabel{NWVU3sy-2Kurrx-1}\nwmargintag{{\nwtagstyle{}\subpageref{NWVU3sy-2Kurrx-1}}}\moddef{Filter for positive integers~{\nwtagstyle{}\subpageref{NWVU3sy-2Kurrx-1}}}\endmoddef\nwstartdeflinemarkup\nwusesondefline{\\{NWVU3sy-2BftRL-1}\\{NWVU3sy-2xVIrD-1}}\nwenddeflinemarkup
IsInt and IsPosInt
\nwused{\\{NWVU3sy-2BftRL-1}\\{NWVU3sy-2xVIrD-1}}\nwendcode{}\nwbegindocs{17}\nwdocspar

\nwenddocs{}\nwbegincode{18}\sublabel{NWVU3sy-2BftRL-1}\nwmargintag{{\nwtagstyle{}\subpageref{NWVU3sy-2BftRL-1}}}\moddef{lib/PerfectNumbers.gd~{\nwtagstyle{}\subpageref{NWVU3sy-2BftRL-1}}}\endmoddef\nwstartdeflinemarkup\nwenddeflinemarkup
#! @Chapter PerfectNumbers

#! @Section The \nwlinkedidentc{IsPerfect}{NWVU3sy-2xVIrD-1}() Operation

#! @Description
#!  Determine whether a positive <A>int</A> is perfect.
#! @Arguments int
DeclareOperation( "\nwlinkedidentc{IsPerfect}{NWVU3sy-2xVIrD-1}",
    [ \LA{}Filter for positive integers~{\nwtagstyle{}\subpageref{NWVU3sy-2Kurrx-1}}\RA{} ] );
\nwnotused{lib/PerfectNumbers.gd}\nwidentuses{\\{{\nwixident{IsPerfect}}{IsPerfect}}}\nwindexuse{\nwixident{IsPerfect}}{IsPerfect}{NWVU3sy-2BftRL-1}\nwendcode{}\nwbegindocs{19}\nwdocspar

\nwenddocs{}\nwbegincode{20}\sublabel{NWVU3sy-2xVIrD-1}\nwmargintag{{\nwtagstyle{}\subpageref{NWVU3sy-2xVIrD-1}}}\moddef{lib/PerfectNumbers.gi~{\nwtagstyle{}\subpageref{NWVU3sy-2xVIrD-1}}}\endmoddef\nwstartdeflinemarkup\nwenddeflinemarkup
#! @Chapter PerfectNumbers

#! @Section The \nwlinkedidentc{IsPerfect}{NWVU3sy-2xVIrD-1}() Operation

\nwindexdefn{\nwixident{IsPerfect}}{IsPerfect}{NWVU3sy-2xVIrD-1}InstallMethod( \nwlinkedidentc{IsPerfect}{NWVU3sy-2xVIrD-1},
    "for a positive integer",
    [ \LA{}Filter for positive integers~{\nwtagstyle{}\subpageref{NWVU3sy-2Kurrx-1}}\RA{} ],
    \LA{}Determine whether a positive integer is perfect, using Sigma~{\nwtagstyle{}\subpageref{NWVU3sy-1Ey6ln-1}}\RA{} );

#! @BeginExample
\LA{}Find all perfect numbers less than 1000~{\nwtagstyle{}\subpageref{NWVU3sy-h8dy6-1}}\RA{}
#! \LA{}All perfect numbers less than 1000~{\nwtagstyle{}\subpageref{NWVU3sy-1gKlN7-1}}\RA{}
#! @EndExample
\nwnotused{lib/PerfectNumbers.gi}\nwidentdefs{\\{{\nwixident{IsPerfect}}{IsPerfect}}}\nwendcode{}\nwbegindocs{21}\nwdocspar


\subsection{Tests}

\nwenddocs{}\nwbegincode{22}\sublabel{NWVU3sy-36UFYc-1}\nwmargintag{{\nwtagstyle{}\subpageref{NWVU3sy-36UFYc-1}}}\moddef{<Run the tests~{\nwtagstyle{}\subpageref{NWVU3sy-36UFYc-1}}}\endmoddef\nwstartdeflinemarkup\nwenddeflinemarkup
gap> Test("tst/PerfectNumbers.tst");
\nwnotused{<Run the tests}\nwendcode{}\nwbegindocs{23}\nwdocspar

\nwenddocs{}\nwbegincode{24}\sublabel{NWVU3sy-2W1Ke-1}\nwmargintag{{\nwtagstyle{}\subpageref{NWVU3sy-2W1Ke-1}}}\moddef{tst/PerfectNumbers.tst~{\nwtagstyle{}\subpageref{NWVU3sy-2W1Ke-1}}}\endmoddef\nwstartdeflinemarkup\nwenddeflinemarkup
gap> START_TEST("AAIG package: PerfectNumbers.tst");
gap> LoadPackage("AAIG", false);
#I  method installed for \nwlinkedidentc{IsPerfect}{NWVU3sy-2xVIrD-1} matches more than one declaration
true
gap> \LA{}Find all perfect numbers less than 1000~{\nwtagstyle{}\subpageref{NWVU3sy-h8dy6-1}}\RA{}
\LA{}All perfect numbers less than 1000~{\nwtagstyle{}\subpageref{NWVU3sy-1gKlN7-1}}\RA{}
gap> STOP_TEST( "PerfectNumbers.tst", 10000 );
\nwnotused{tst/PerfectNumbers.tst}\nwidentuses{\\{{\nwixident{IsPerfect}}{IsPerfect}}}\nwindexuse{\nwixident{IsPerfect}}{IsPerfect}{NWVU3sy-2W1Ke-1}\nwendcode{}

\nwixlogsorted{c}{{<Run the tests}{NWVU3sy-36UFYc-1}{\nwixd{NWVU3sy-36UFYc-1}}}%
\nwixlogsorted{c}{{All perfect numbers less than 1000}{NWVU3sy-1gKlN7-1}{\nwixd{NWVU3sy-1gKlN7-1}\nwixu{NWVU3sy-2xVIrD-1}\nwixu{NWVU3sy-2W1Ke-1}}}%
\nwixlogsorted{c}{{Compute the aliquot sum of a positive integer}{NWVU3sy-xApnh-1}{\nwixd{NWVU3sy-xApnh-1}}}%
\nwixlogsorted{c}{{Determine whether a positive integer is perfect}{NWVU3sy-3gU79I-1}{\nwixd{NWVU3sy-3gU79I-1}}}%
\nwixlogsorted{c}{{Determine whether a positive integer is perfect, using Sigma}{NWVU3sy-1Ey6ln-1}{\nwixd{NWVU3sy-1Ey6ln-1}\nwixu{NWVU3sy-2xVIrD-1}}}%
\nwixlogsorted{c}{{Euclid's IX.36}{NWVU3sy-hWGw1-1}{\nwixd{NWVU3sy-hWGw1-1}}}%
\nwixlogsorted{c}{{Filter for positive integers}{NWVU3sy-2Kurrx-1}{\nwixd{NWVU3sy-2Kurrx-1}\nwixu{NWVU3sy-2BftRL-1}\nwixu{NWVU3sy-2xVIrD-1}}}%
\nwixlogsorted{c}{{Find all perfect numbers less than 1000}{NWVU3sy-h8dy6-1}{\nwixd{NWVU3sy-h8dy6-1}\nwixu{NWVU3sy-2xVIrD-1}\nwixu{NWVU3sy-2W1Ke-1}}}%
\nwixlogsorted{c}{{lib/PerfectNumbers.gd}{NWVU3sy-2BftRL-1}{\nwixd{NWVU3sy-2BftRL-1}}}%
\nwixlogsorted{c}{{lib/PerfectNumbers.gi}{NWVU3sy-2xVIrD-1}{\nwixd{NWVU3sy-2xVIrD-1}}}%
\nwixlogsorted{c}{{Not all perfect}{NWVU3sy-2uZKeV-1}{\nwixd{NWVU3sy-2uZKeV-1}}}%
\nwixlogsorted{c}{{tst/PerfectNumbers.tst}{NWVU3sy-2W1Ke-1}{\nwixd{NWVU3sy-2W1Ke-1}}}%
\nwixlogsorted{i}{{\nwixident{AliquotSum}}{AliquotSum}}%
\nwixlogsorted{i}{{\nwixident{IsPerfect}}{IsPerfect}}%
\nwbegindocs{25}\nwdocspar

\chapter{Chunks}
\nowebchunks

\chapter{Index}
\nowebindex

\bibliography{aaig}
\bibliographystyle{plainnat}

\end{document}
\nwenddocs{}
